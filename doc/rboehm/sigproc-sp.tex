% THIS IS SIGPROC-SP.TEX - VERSION 3.1
% WORKS WITH V3.2SP OF ACM_PROC_ARTICLE-SP.CLS
% APRIL 2009
%
% It is an example file showing how to use the 'acm_proc_article-sp.cls' V3.2SP
% LaTeX2e document class file for Conference Proceedings submissions.
% ----------------------------------------------------------------------------------------------------------------
% This .tex file (and associated .cls V3.2SP) *DOES NOT* produce:
%       1) The Permission Statement
%       2) The Conference (location) Info information
%       3) The Copyright Line with ACM data
%       4) Page numbering
% ---------------------------------------------------------------------------------------------------------------
% It is an example which *does* use the .bib file (from which the .bbl file
% is produced).
% REMEMBER HOWEVER: After having produced the .bbl file,
% and prior to final submission,
% you need to 'insert'  your .bbl file into your source .tex file so as to provide
% ONE 'self-contained' source file.
%
% Questions regarding SIGS should be sent to
% Adrienne Griscti ---> griscti@acm.org
%
% Questions/suggestions regarding the guidelines, .tex and .cls files, etc. to
% Gerald Murray ---> murray@hq.acm.org
%
% For tracking purposes - this is V3.1SP - APRIL 2009

\documentclass{acm_proc_article-sp}

\begin{document}

\title{Prototype of an event-driven Future Internet Cockpit -  Real-Time Sensor Events for Complex Event Processing}
% \subtitle{[Event processing]}

%
% You need the command \numberofauthors to handle the 'placement
% and alignment' of the authors beneath the title.
%
% For aesthetic reasons, we recommend 'three authors at a time'
% i.e. three 'name/affiliation blocks' be placed beneath the title.
%
% NOTE: You are NOT restricted in how many 'rows' of
% "name/affiliations" may appear. We just ask that you restrict
% the number of 'columns' to three.
%
% Because of the available 'opening page real-estate'
% we ask you to refrain from putting more than six authors
% (two rows with three columns) beneath the article title.
% More than six makes the first-page appear very cluttered indeed.
%
% Use the \alignauthor commands to handle the names
% and affiliations for an 'aesthetic maximum' of six authors.
% Add names, affiliations, addresses for
% the seventh etc. author(s) as the argument for the
% \additionalauthors command.
% These 'additional authors' will be output/set for you
% without further effort on your part as the last section in
% the body of your article BEFORE References or any Appendices.

\numberofauthors{1} %  in this sample file, there are a *total*
% of EIGHT authors. SIX appear on the 'first-page' (for formatting
% reasons) and the remaining two appear in the \additionalauthors section.
%
\author{
% You can go ahead and credit any number of authors here,
% e.g. one 'row of three' or two rows (consisting of one row of three
% and a second row of one, two or three).
%
% The command \alignauthor (no curly braces needed) should
% precede each author name, affiliation/snail-mail address and
% e-mail address. Additionally, tag each line of
% affiliation/address with \affaddr, and tag the
% e-mail address with \email.
%
% 1st. author
\alignauthor
Robin B\"ohm\\
       \affaddr{University of Duisburg Essen}\\
       \affaddr{Sch\"utzenbahn 70, 45117, Essen, Germany}\\
       \email{robin.boehm@stud.uni-due.de}
}

\maketitle
\begin{abstract}

\begin{itemize}
	\item abstract
\end{itemize}

\end{abstract}

\section{Introduction}

\begin{itemize}
	\item aim / contribution / motivation
		\begin{itemize}
			\item logistic cockpit prototype
			\item Future Internet
			\item Complex Event Processing
			\item Business Process management in logistic sector
			\item goods became "unmarketable"
			\item decrease the latency -> McKinsy Hackathron
			\item improvement here through EDA
		\end{itemize}
	\item Event Driven Architecture
		\begin{itemize}
			\item what
			\item why in this problem domain so valuable 
			\item value of real time events
		\end{itemize}
	\item Internet of Things
		\begin{itemize}
			\item detailed view on real world
			\item sensors
		\end{itemize}
	\item What is CEP
		\begin{itemize}
			\item Sensor Input from IoT
			\item Rules
			\item Complex Events
		\end{itemize}
	\item Scenario
	\item Prototype
			\begin{itemize}
				\item EDA
				\item COSM
				\item Esper
				\item HTML5 WebSocket
				\item multi-device support
			\end{itemize}			
		
\item structure of this document								 
\end{itemize}




\section{Basic}
\begin{itemize}
	\item Intro
	\begin{itemize}	
	\item Hackathron, latency
	\end{itemize}
	\item Technologies
	\begin{itemize}
	\item Future Internet
	\item Internet of Service
	\item Internet of Things
	\item Cosm
	\end{itemize}
	
	\item Event-Driven Architecture	
	\begin{itemize}
	\item Desc
	\item Other Arch
	\item Websocket	
	\item Complex Event Processing	
	\item Esper
	\item cockpit?
	\end{itemize}

\end{itemize}


\section{Related Work}

TODO!!

\section{Use Case}

\begin{itemize}
	\item Problem
	\begin{itemize}
		\item logistic industry
		\item perishable goods in trucks
		\item planing routes
		\item alternatives
		\item planing risk
	\end{itemize}
	
	\item Solution approach
	\begin{itemize}
		\item sensors
		\item cockpit
		\item different devices
		\item instant notifications
		\item sensor level / rating
	\end{itemize}
	
\end{itemize}



\section{Implementation}
\begin{itemize}
	\item Introduction
	\item Diagrams
	\item Real-Time event monitoring
	\item Short Desc
	\begin{itemize}
		\item COSM
		\begin{itemize}
			\item Websocket Protokol
			\item API Key
			\item Objekt Parsing
			\item Route Model as COSM Sensors
			\item Data Manipulation / Event Generation
			\item Push To Esper
		\end{itemize}
		\item Esper
		\begin{itemize}
			\item General Concept
			\item Complex Events
			\item Rules
			\item Listeners
		\end{itemize}
		\item JS Frontend
		\begin{itemize}
			\item Websocket Event Model
			\item Short Desc Components
			\item Google Maps Route
			\item jQuery Table visualization
		\end{itemize}
	\end{itemize}
\end{itemize}


\section{Conclusions}
\begin{itemize}
	\item FI
	\item State-of-the art technologies
	\item real time events
	\item Logistic Information System / Business Process Management
	\item Multi-Device Support HTML5
\end{itemize}
%\end{document}  % This is where a 'short' article might terminate

%
% The following two commands are all you need in the
% initial runs of your .tex file to
% produce the bibliography for the citations in your paper.
\bibliographystyle{abbrv}
\bibliography{sigproc}  % sigproc.bib is the name of the Bibliography in this case
% You must have a proper ".bib" file
%  and remember to run:
% latex bibtex latex latex
% to resolve all references
%
% ACM needs 'a single self-contained file'!
%
%APPENDICES are optional
%\balancecolumns
\appendix
%Appendix A
\end{document}
